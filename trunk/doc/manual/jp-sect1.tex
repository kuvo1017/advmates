%#!platex AdvMates-0.5-manual-jp.tex

\section{はじめに}
\label{sect:introduction}

本書はADVENTURE Project\cite{adv:HP}において開発中のマルチエージェント交通流シミュレータADVENTURE\_Mates (略称MATES[メイツ]) の使用マニュアルである.

本章ではADVENTURE\_Matesの概要および実行までの操作手順を説明する.

\subsection{動作環境}
\label{subsect:environment}

本プログラムのコンパイルに必要なものは,GNU CコンパイラおよびC++コンパイラ (gcc,g++) ,CMake,makeである.システムにzlibがインストールされていれば利用できる.可視化およびGUIを伴うプログラム (advmates-sim) のコンパイルにはOpenGLおよびGTK+2が必要である.

本プログラムはLinuxにて動作確認を行っているが,多くの機能はWindows上に構築されたCygwinあるいはMinGW/MSYSでも実行可能である.Windows上での可視化およびGUIを伴うプログラム (advmates-sim) のコンパイルにはOpenGLが必要である.

なお,プレ処理モジュールではQt (バージョン4以降) を使用している.プレ処理モジュールの詳細についてはそれぞれのマニュアルを参照されたい.

\subsection{コンパイルとインストール}
\label{subsect:compiling}

ADVENTURE\_Matesのモジュール群をコンパイル,インストールするには以下の手順に従う.

\newcounter{buildnum}
\begin{enumerate}
  \renewcommand{\labelenumi}{(\arabic{enumi})}

\item アーカイブファイルの展開
  \setcounter{buildnum}{\value{enumi}}
\end{enumerate}

ダウンロードしたファイルのあるディレクトリで以下を実行する.

\begin{screen}
  \% tar xzf AdvMates-{\advmatesver}.tar.gz
\end{screen}

ただし,``\%''はコマンドプロンプトを表すため,実際には入力する必要はない.

アーカイブファイルの展開により,AdvMates-{\advmatesver}ディレクトリが作成される.また,AdvMates-{\advmatesver}ディレクトリは次のサブディレクトリを含んでいる.

\begin{tabbing}
  \hspace{1em}\=\hspace{10em}\=\hfill\kill
  \>- doc     \>: ドキュメント類\\
  \>- examples\>: サンプルデータ\\
  \>- include \>: ヘッダファイル群\\
  \>- lib     \>: ライブラリ群\\
  \>- pre     \>: プレ処理モジュールのソースファイル\\
  \>- solver  \>: メインモジュールのソースファイル\\
\end{tabbing}

\begin{enumerate}
  \setcounter{enumi}{\value{buildnum}}
  \renewcommand{\labelenumi}{(\arabic{enumi})}
\item Makefileの生成
  \setcounter{buildnum}{\value{enumi}}
\end{enumerate}

AdvMates-{\advmatesver}ディレクトリにて,

\begin{screen}
  \% cmake
\end{screen}

を実行する.ただし,MSYS上でコンパイルする場合は

\begin{screen}
  \% cmake -G ``MSYS Makefiles''
\end{screen}

とする.\\

\begin{enumerate}
  \setcounter{enumi}{\value{buildnum}}
  \renewcommand{\labelenumi}{(\arabic{enumi})}
\item コンパイル
  \setcounter{buildnum}{\value{enumi}}
\end{enumerate}

AdvMates-{\advmatesver}ディレクトリにて,

\begin{screen}
  \% make
\end{screen}

を実行する.\\

\begin{enumerate}
  \setcounter{enumi}{\value{buildnum}}
  \renewcommand{\labelenumi}{(\arabic{enumi})}
\item インストール
  \setcounter{buildnum}{\value{enumi}}
\end{enumerate}

コンパイルに成功したら,以下のコマンドによりインストールを行う.

\begin{screen}
  \% make install
\end{screen}

ただし,インストール先ディレクトリに書き込み権限を持ったユーザによって行う必要がある.以上の操作により,以下のファイルが指定されたディレクトリにインストールされる.

\subsection{実行方法}
\label{subsect:execution}

ADVENTURE\_Matesにはadvmates-calcとadvmates-simという2つの実行モジュールがある.

\newcounter{execnum}
\begin{enumerate}
  \renewcommand{\labelenumi}{(\arabic{enumi})}
\item advmates-calc
  \setcounter{execnum}{\value{enumi}}
\end{enumerate}

画像による出力なしのシミュレーションを実行する.デフォルトで出力ファイルを書き込む.1時間程度のシミュレーションの時系列データを出力すると1GBを超えることがあるので,ディスク容量には注意する必要がある.

advmates-calcは以下のコマンドで実行する.実行時には表\ref{tbl:advcalc-option}に示すオプションを指定できる.

\begin{screen}
  \% advmates-calc [オプション]
\end{screen}

\begin{table}[hbt]
  \caption{advmates-calcの実行時オプション}
  \small
  \begin{center}
    \begin{tabular}{|l|p{20em}|} \hline
      \multicolumn{1}{|c|}{オプション}
      & \multicolumn{1}{|c|}{説明} \\ \hline\hline
      
      -d {\sl DataDir}
      & 入出力ディレクトリのルートパスを{\sl DataDir}に指定する.
      指定しない場合はカレントディレクトリをルートパスとする. \\ \hline
      
      -r {\sl Number}
      & 乱数の種を{\sl Number}に指定する.
      デフォルトでは時刻をもとに与えられる. \\ \hline
      
      --no-verbose
      & 詳細な画面表示を省略する.\\ \hline
      
      --no-input
      & 入力ファイルなしでテストデータを利用する.以下の--no-input-mapおよび--no-input-signalが同時に指定された場合と同じ挙動となる.\\ \hline
      
      --no-input-map
      & 入力ファイルのうち,道路ネットワークデータ(network.txtおよびmapPosition.txt)なしでテストデータを利用する.\\ \hline
      
      --no-input-signal
      & signalsディレクトリ以下の信号入力ファイルを読み込まず,すべての信号が青であるとみなす.\\ \hline
      
      --no-generate-random-vehicle
      & 入力ファイル指定された車両以外を生成しない.\\ \hline
      
      -t {\sl MaxTime}
      & 計算対象となるシミュレーション内の時間を{\sl MaxTime}[msec]に指定する.指定しない場合は3,600,000[msec]=1時間である.グローバル変数が指定されていればそれを優先する.\\ \hline
      
      -s
      & 時系列データを出力しない.\\ \hline
      
      -m
      & 車両感知器による計測データを出力しない.\\ \hline
      
      --no-output-monitor-d
      & 車両感知器による計測データのうち,詳細出力 (detD****.txt) を出力しない.\\ \hline
      
      --no-output-monitor-s
      & 車両感知器による計測データのうち,統計出力 (detS****.txt) を出力しない.\\ \hline
      
      -g
      & 車両発生カウンタが計測するデータを出力しない.\\ \hline
      
      --help
      & ヘルプを表示する.\\ \hline
      
    \end{tabular}
  \end{center}
  \label{tbl:advcalc-option}
\end{table}

\begin{enumerate}
  \setcounter{enumi}{\value{execnum}}
  \renewcommand{\labelenumi}{(\arabic{enumi})}
\item advmates-sim
  \setcounter{execnum}{\value{enumi}}
\end{enumerate}

画像による出力およびGUI付きのシミュレーションを実行する.GUIから指定された場合のみ出力ファイルを書き込む.GUIに関しては後述する.

advmates-simは以下のコマンドで実行する.実行時には表\ref{tbl:advsim-option}に示すオプションを指定できる.

\begin{screen}
  \% advmates-sim [オプション]
\end{screen}

\begin{table}[hbt]
  \caption{advmates-simの実行時オプション}
  \small
  \begin{center}
    \begin{tabular}{|l|p{20em}|} \hline
      \multicolumn{1}{|c|}{オプション}
      & \multicolumn{1}{|c|}{説明} \\ \hline\hline
      
      -d {\sl DataDir}
      & 入出力ディレクトリのルートパスを{\sl DataDir}に指定する.
      指定しない場合はカレントディレクトリをルートパスとする. \\ \hline
      
      -r {\sl Number}
      & 乱数の種を{\sl Number}に指定する.
      デフォルトでは時刻をもとに与えられる. \\ \hline
      
      --no-verbose
      & 詳細な画面表示を省略する.\\ \hline
      
      --no-input
      & 入力ファイルなしでテストデータを利用する.以下の--no-input-mapおよび--no-input-signalが同時に指定された場合と同じ挙動となる.\\ \hline
      
      --no-input-map
      & 入力ファイルのうち,道路ネットワークデータ(network.txtおよびmapPosition.txt)なしでテストデータを利用する.\\ \hline
      
      --no-input-signal
      & signalsディレクトリ以下の信号入力ファイルを読み込まず,すべての信号が青であるとみなす.\\ \hline
      
      --no-generate-random-vehicle
      & 入力ファイル指定された車両以外を生成しない.\\ \hline
      
      --help
      & ヘルプを表示する.\\ \hline
      
    \end{tabular}
  \end{center}
  \label{tbl:advsim-option}
\end{table}
